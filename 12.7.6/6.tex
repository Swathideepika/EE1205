\let\negmedspace\undefined
\let\negthickspace\undefined
\documentclass[journal,12pt,twocolumn]{IEEEtran}

\usepackage{cite}
\usepackage{amsmath,amssymb,amsfonts,amsthm}
\usepackage{algorithmic}
\usepackage{graphicx}
\usepackage{textcomp}
\usepackage{xcolor}
\usepackage{txfonts}
\usepackage{listings}
\usepackage{enumitem}
\usepackage{mathtools}
\usepackage{gensymb}
\usepackage[breaklinks=true]{hyperref}
\usepackage{tkz-euclide} % loads  TikZ and tkz-base
\usepackage{listings}
\usepackage{circuitikz}
\usepackage{graphicx}

%\newcounter{MYtempeqncnt}
\DeclareMathOperator*{\Res}{Res}
%\renewcommand{\baselinestretch}{2}
\renewcommand\thesection{\arabic{section}}
\renewcommand\thesubsection{\thesection.\arabic{subsection}}
\renewcommand\thesubsubsection{\thesubsection.\arabic{subsubsection}}

\renewcommand\thesectiondis{\arabic{section}}
\renewcommand\thesubsectiondis{\thesectiondis.\arabic{subsection}}
\renewcommand\thesubsubsectiondis{\thesubsectiondis.\arabic{subsubsection}}

% correct bad hyphenation here
\hyphenation{op-tical net-works semi-conduc-tor}
\def\inputGnumericTable{}                                 %%

\lstset{
	frame=single,
	breaklines=true,
	columns=fullflexible
}



\newtheorem{theorem}{Theorem}[section]
\newtheorem{problem}{Problem}
\newtheorem{proposition}{Proposition}[section]
\newtheorem{lemma}{Lemma}[section]
\newtheorem{corollary}[theorem]{Corollary}
\newtheorem{example}{Example}[section]
\newtheorem{definition}[problem]{Definition}
\newcommand{\BEQA}{\begin{eqnarray}}
	\newcommand{\EEQA}{\end{eqnarray}}
\newcommand{\define}{\stackrel{\triangle}{=}}
\newcommand\figref{Fig.~\ref}
\newcommand\tabref{Table~\ref}
\bibliographystyle{IEEEtran}
%\bibliographystyle{ieeetr}


\providecommand{\mbf}{\mathbf}
\providecommand{\pr}[1]{\ensuremath{\Pr\left(#1\right)}}
\providecommand{\qfunc}[1]{\ensuremath{Q\left(#1\right)}}
\providecommand{\sbrak}[1]{\ensuremath{{}\left[#1\right]}}
\providecommand{\lsbrak}[1]{\ensuremath{{}\left[#1\right.}}
\providecommand{\rsbrak}[1]{\ensuremath{{}\left.#1\right]}}
\providecommand{\brak}[1]{\ensuremath{\left(#1\right)}}
\providecommand{\lbrak}[1]{\ensuremath{\left(#1\right.}}
\providecommand{\rbrak}[1]{\ensuremath{\left.#1\right)}}
\providecommand{\cbrak}[1]{\ensuremath{\left\{#1\right\}}}
\providecommand{\lcbrak}[1]{\ensuremath{\left\{#1\right.}}
\providecommand{\rcbrak}[1]{\ensuremath{\left.#1\right\}}}
\theoremstyle{remark}
\newtheorem{rem}{Remark}
\newcommand{\sgn}{\mathop{\mathrm{sgn}}}
\providecommand{\abs}[1]{\left\vert#1\right\vert}
\providecommand{\res}[1]{\Res\displaylimits_{#1}}
\providecommand{\norm}[1]{\left\lVert#1\right\rVert}
%\providecommand{\norm}[1]{\lVert#1\rVert}
\providecommand{\mtx}[1]{\mathbf{#1}}
\providecommand{\mean}[1]{E\left[ #1 \right]}
\providecommand{\fourier}{\overset{\mathcal{F}}{ \rightleftharpoons}}
%\providecommand{\hilbert}{\overset{\mathcal{H}}{ \rightleftharpoons}}
\providecommand{\system}{\overset{\mathcal{H}}{ \longleftrightarrow}}
%\newcommand{\solution}[2]{\textbf{Solution:}{#1}}
\newcommand{\solution}{\noindent \textbf{Solution: }}
\newcommand{\cosec}{\,\text{cosec}\,}
\providecommand{\dec}[2]{\ensuremath{\overset{#1}{\underset{#2}{\gtrless}}}}
\newcommand{\myvec}[1]{\ensuremath{\begin{pmatrix}#1\end{pmatrix}}}
\newcommand{\mydet}[1]{\ensuremath{\begin{vmatrix}#1\end{vmatrix}}}
\renewcommand{\abstractname}{Question}

\let\vec\mathbf

	
	\vspace{3cm}
	
	


\newcommand{\permcomb}[4][0mu]{{{}^{#3}\mkern#1#2_{#4}}}
\newcommand{\comb}[1][-1mu]{\permcomb[#1]{C}}

%\IEEEpeerreviewmaketitle

\newcommand \tab [1][1cm]{\hspace*{#1}}
%\newcommand{\Var}{$\sigma ^2$}
\usepackage{amssymb}
\usepackage{amsmath}
\title{
	
\title{NCERT Physics 12.7 Q6}
\author{EE23BTECH11212 - SWATHI DEEPIKA$^{*}$% <-this % stops a space
}


}
\begin{document}

\maketitle

\textbf{Question:} 
Obtain the resonant frequency of a series LCR circuit with $L = 2.0\, H$, $C = 32\, \mu F$, and $R = 10\, \Omega$. What is the Q-value of the circuit.\\

\begin{figure}[h]
	\centering
	
\begin{circuitikz}
    % Define components with values
    \draw (0,0) to [L, l=$2.0\, H$] (2,0)  % Inductor L
    to [R, l=$10\, \Omega$] (4,0)  % Resistor R
    to [C, l=$32\, \mu\mathrm{F}$] (4,-2)  % Capacitor C
    -- (0,-2) to (0,0);  % Connect back to the starting point

    % Add labels
    \node at (1,0.3) {$+$};
    \node at (1,-2.3) {$-$};
\end{circuitikz}

	\caption{LCR Circuit}
	\label{fig:2}
\end{figure}
     
\textbf{Solution: }
In Figure~\figref{fig:2} the following information is provided:
 
 

 \begin{table}[h]
 	\centering
 	\resizebox{6 cm}{!}{
 		\begin{tabular}{|c|c|c|}
    \hline
     \textbf{Symbol} & \textbf{Remarks} \\
    \hline
     $x(0)$ &  $a$\\[6pt]
    \hline 
     $x(1)$ &  $b$\\[6pt]
    \hline
     $x(2)$ &  $c$\\[6pt]
    \hline
     $x(3)$ &  $d$\\[6pt]
    \hline
     $x(4)$ &  $e$\\[6pt]
    \hline
\end{tabular}


 	}
 	\vspace{6 pt}
 	\caption{Parameters}
 	\label{tab:my_label} 
 \end{table}
 

\section*{Series LCR Circuit Analysis}

Now, the voltage transfer function (\(\frac{V(s)}{I(s)} = H(s)\)) is given by Ohm's Law in the Laplace domain:

\begin{align}
\frac{V(s)}{I(s)} = R + sL + \frac{1}{sC} 
\end{align}

Now, after substitution the equation is 

\begin{align}
\frac{V(s)}{I(s)} = 10 + 2s + \frac{1}{32 \times 10^{-6}s}
\end{align}

This is the voltage transfer function for the series LCR circuit in the Laplace domain.

\textbf{Resonant Frequency (\(\omega_0\)):}

At resonance, the impedance is purely resistive, meaning the imaginary part of \(Z(s)\) is zero.

\begin{align}
\text{Im}\{Z(s)\} = \omega L - \frac{1}{\omega C} = 0
\end{align}
Solving for \(\omega_0\):

\begin{align}
\omega_0 = \frac{1}{\sqrt{LC}}
\end{align}

Substituting values:

\begin{align}
\omega_0 = \frac{1}{\sqrt{2 \, \text{H} \times 32 \, \mu\text{F}}} = 125 \, \text{rad/s}
\end{align}

\textbf{Quality Factor (Q) Calculation:}

\begin{align}
\frac{\omega_r}{\Delta \omega} = \frac{\omega_0 L}{R}
\end{align}

\begin{align}
Q = \frac{\omega_0 L}{R} = \frac{125 \, \text{rad/s} \times 2 \, \text{H}}{10 \, \Omega} = 25
\end{align}

Therefore, the quality factor of the LCR circuit is 25.


The equivalent s domain of the circuit is :

\begin{figure}[h]
 \centering
    \input{figs/fig2}
    \caption{LCR Circuit in s-domain}
    
\end{figure}

\end{document}



