\let\negmedspace\undefined
\let\negthickspace\undefined
\documentclass[journal,12pt,twocolumn]{IEEEtran}

\usepackage{cite}
\usepackage{amsmath,amssymb,amsfonts,amsthm}
\usepackage{algorithmic}
\usepackage{graphicx}
\usepackage{textcomp}
\usepackage{xcolor}
\usepackage{txfonts}
\usepackage{listings}
\usepackage{enumitem}
\usepackage{mathtools}
\usepackage{gensymb}
\usepackage[breaklinks=true]{hyperref}
\usepackage{tkz-euclide} % loads  TikZ and tkz-base
\usepackage{listings}
\usepackage{circuitikz}
\usepackage{graphicx}

%\newcounter{MYtempeqncnt}
\DeclareMathOperator*{\Res}{Res}
%\renewcommand{\baselinestretch}{2}
\renewcommand\thesection{\arabic{section}}
\renewcommand\thesubsection{\thesection.\arabic{subsection}}
\renewcommand\thesubsubsection{\thesubsection.\arabic{subsubsection}}

\renewcommand\thesectiondis{\arabic{section}}
\renewcommand\thesubsectiondis{\thesectiondis.\arabic{subsection}}
\renewcommand\thesubsubsectiondis{\thesubsectiondis.\arabic{subsubsection}}

% correct bad hyphenation here
\hyphenation{op-tical net-works semi-conduc-tor}
\def\inputGnumericTable{}                                 %%

\lstset{
	frame=single,
	breaklines=true,
	columns=fullflexible
}



\newtheorem{theorem}{Theorem}[section]
\newtheorem{problem}{Problem}
\newtheorem{proposition}{Proposition}[section]
\newtheorem{lemma}{Lemma}[section]
\newtheorem{corollary}[theorem]{Corollary}
\newtheorem{example}{Example}[section]
\newtheorem{definition}[problem]{Definition}
\newcommand{\BEQA}{\begin{eqnarray}}
	\newcommand{\EEQA}{\end{eqnarray}}
\newcommand{\define}{\stackrel{\triangle}{=}}
\newcommand\figref{Fig.~\ref}
\newcommand\tabref{Table~\ref}
\bibliographystyle{IEEEtran}
%\bibliographystyle{ieeetr}


\providecommand{\mbf}{\mathbf}
\providecommand{\pr}[1]{\ensuremath{\Pr\left(#1\right)}}
\providecommand{\qfunc}[1]{\ensuremath{Q\left(#1\right)}}
\providecommand{\sbrak}[1]{\ensuremath{{}\left[#1\right]}}
\providecommand{\lsbrak}[1]{\ensuremath{{}\left[#1\right.}}
\providecommand{\rsbrak}[1]{\ensuremath{{}\left.#1\right]}}
\providecommand{\brak}[1]{\ensuremath{\left(#1\right)}}
\providecommand{\lbrak}[1]{\ensuremath{\left(#1\right.}}
\providecommand{\rbrak}[1]{\ensuremath{\left.#1\right)}}
\providecommand{\cbrak}[1]{\ensuremath{\left\{#1\right\}}}
\providecommand{\lcbrak}[1]{\ensuremath{\left\{#1\right.}}
\providecommand{\rcbrak}[1]{\ensuremath{\left.#1\right\}}}
\theoremstyle{remark}
\newtheorem{rem}{Remark}
\newcommand{\sgn}{\mathop{\mathrm{sgn}}}
\providecommand{\abs}[1]{\left\vert#1\right\vert}
\providecommand{\res}[1]{\Res\displaylimits_{#1}}
\providecommand{\norm}[1]{\left\lVert#1\right\rVert}
%\providecommand{\norm}[1]{\lVert#1\rVert}
\providecommand{\mtx}[1]{\mathbf{#1}}
\providecommand{\mean}[1]{E\left[ #1 \right]}
\providecommand{\fourier}{\overset{\mathcal{F}}{ \rightleftharpoons}}
%\providecommand{\hilbert}{\overset{\mathcal{H}}{ \rightleftharpoons}}
\providecommand{\system}{\overset{\mathcal{H}}{ \longleftrightarrow}}
%\newcommand{\solution}[2]{\textbf{Solution:}{#1}}
\newcommand{\solution}{\noindent \textbf{Solution: }}
\newcommand{\cosec}{\,\text{cosec}\,}
\providecommand{\dec}[2]{\ensuremath{\overset{#1}{\underset{#2}{\gtrless}}}}
\newcommand{\myvec}[1]{\ensuremath{\begin{pmatrix}#1\end{pmatrix}}}
\newcommand{\mydet}[1]{\ensuremath{\begin{vmatrix}#1\end{vmatrix}}}
\renewcommand{\abstractname}{Question}

\let\vec\mathbf

	
	\vspace{3cm}
	
	


\newcommand{\permcomb}[4][0mu]{{{}^{#3}\mkern#1#2_{#4}}}
\newcommand{\comb}[1][-1mu]{\permcomb[#1]{C}}

%\IEEEpeerreviewmaketitle

\newcommand \tab [1][1cm]{\hspace*{#1}}
%\newcommand{\Var}{$\sigma ^2$}
\usepackage{amssymb}
\usepackage{amsmath}
\title{
	
\title{NCERT Discrete 11.5.9 Q20}
\author{EE23BTECH11061 - SWATHI DEEPIKA$^{*}$% <-this % stops a space
}


}
\begin{document}

\maketitle

\textbf{Question:} 
If $a$,$b$,$c$ are in A.P.;$b$,$c$,$d$ are in G.P and $\frac{1}{c}$, $\frac{1}{d}$, $\frac{1}{e}$ are in A.P. prove that $a$,$c$,$e$ are in G.P.
\solution
 \begin{table}[h]
 	\centering
 	\resizebox{6 cm}{!}{
 		
    \begin{tabular}{|c|c|c|}
    \hline
     \textbf{Symbol} & \textbf{Value} &
     \textbf{Description}\\
    \hline
     $x(n)$ &  $(4n+1)u(n)$ & The nth term of the sequence\\[6pt]
    \hline 
     $x(17)$ &  $?$ & 17nth term \\[6pt]
    \hline
     $x(24)$ &  $?$ & 24th term\\[6pt]
    \hline
     
\end{tabular}

 	}
 	\vspace{6 pt}
 	\caption{Parameters}
 	\label{tab:swa_tabel} 
 \end{table}

 
\begin{align}
b-a = c-b\\
2b=a+c \label{eq: sw1}
\end{align}
\begin{align}
c^2 = b\times d
\end{align}
\begin{align}
d= \frac{c^2}{b} \label{eq: sw2}
\end{align}
\begin{align}
\frac{1}{d} - \frac{1}{c} = \frac{1}{e} - \frac{1}{d}\\
\frac{2}{d} = \frac{1}{c} + \frac{1}{e}
\end{align}
From \eqref{eq: sw2},
\begin{align}
\frac{2b}{c^2} = \frac{1}{c} + \frac{1}{e}
\end{align}

From \eqref{eq: sw1},
\begin{align}\frac{a + c}{c^2} = \frac{1}{c} + \frac{1}{e}\\
\frac{a}{c^2} + \frac{1}{c} = \frac{1}{c} + \frac{1}{e}\\
a \times e = {c}^2\\
{y(1)}^2 = y(0) \times y(2)
\end{align}

So, $y(0)$,$y(1)$,$y(2)$ are in G.P\\

\begin{enumerate}

\item For $y(n)$:
    \begin{align}
        y(n) &= y(0)\left(\frac{y(1)}{y(0)}\right)^n u(n)
    \end{align}
    \begin{equation*}
     y(n) \longleftrightarrow Y(z)
     \end{equation*}
    \begin{align}
        Y(z) &= \frac{y(1)}{1-\frac{y(1)}{y(0)}z^{-1}}, \quad \left|z\right|>\left|\frac{y(1)}{y(0)}\right|
    \end{align}
    
    \item For $x_1(n)$:
    \begin{align}
        x_1(n) &= (x_1(0) + n(x_1(1)-x_1(0)))u(n)
    \end{align}
    \begin{equation*}
    x_1(n) \longleftrightarrow X_1(z)
    \end{equation*}
    \begin{align}
        X_1(z) &= \frac{x_1(0)}{1-z^{-1}} + \frac{(x_1(1)-x_1(0))z^{-1}}{(1-z^{-1})^2} , \quad |z| > 1
    \end{align}

    \item For $x_2(n)$:
    \begin{align}
        x_2(n) &= x_2(0)\left(\frac{x_2(1)}{x_2(0)}\right)^n u(n)
    \end{align}
    \begin{equation*}
     x_2(n) \longleftrightarrow X_2(z)
     \end{equation*}
    \begin{align}
        X_2(z) &= \frac{x_2(1)}{1-\frac{x_2(1)}{x_2(0)}z^{-1}},  \quad \left|z\right|>\left|\frac{x_2(1)}{x_2(0)}\right|
    \end{align}

    \item For $x_3(n)$:
    \begin{align}
        x_3(n) &= \left(x_3(0) + n\left(x_3(0) - x_3(1)\right)\right)u(n)
    \end{align}
    \begin{equation*}
   x_3(n) \longleftrightarrow X_3(z)
   \end{equation*}
    \begin{align}
        X_3(z) &= \frac{x_3(0)}{1-z^{-1}} + \left(x_3(1) - x_3(0)\right)\frac{z^{-1}}{(1-z^{-1})^2} , \quad |z| > 1
    \end{align}
\end{enumerate}


\end{document}





